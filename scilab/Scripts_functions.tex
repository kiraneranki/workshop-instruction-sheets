\documentclass[17pt]{beamer}
\usepackage{amsmath}
\usepackage{hyperref}
\usepackage{epsfig}
%\definecolor{Purple1}{RGB}{0,102,102}
\setbeamercolor{structure}{fg=brown}
%\definecolor{blue}{RGB}{1,51,102}

\setbeamercolor{alerted text}{fg=brown}
\usepackage{verbatim}
\usepackage{bm}
\usepackage{txfonts}
\newenvironment{colorverbatim}[1][]
{
\color{blue}
}
{
\endverbatim
}
\usepackage{beamerthemesplit}
\usepackage{graphicx}
\usepackage{eso-pic}
\usepackage{beamerthemeshadow}
\beamertemplateshadingbackground{blue!5}{yellow!10}
\usepackage{beamerthemesplit}
\logo{\includegraphics[height=1cm]{3t-logo.pdf}}

\begin{document}
\sffamily 
\bfseries
\title
[ Scripts and Functions
\hspace{0.5cm}
\insertframenumber/\inserttotalframenumber]
{\normalsize Scripts and Functions}
\vspace{-1 cm}
\author[ Script \& Narration-Anuradha Amrutkar]{{ \small Talk to a Teacher \\ National Mission on Education through ICT \\ http://spoken-tutorial.org }\\{\scriptsize Script \& Narration} \vspace{-0.2cm}\\  {\small Anuradha Amrutkar}\vspace{-0.3cm} \\{\scriptsize (IIT Bombay)}\vspace{-1 cm}}

\date{ \scriptsize 19 August 2011}

\begin{frame}
\maketitle
\end{frame}

\begin{frame}[fragile]
\frametitle{Script Files}
Let us start with a brief introduction to the file formats in Scilab\pause
\begin{itemize}
\item<+-|alert@+>  When several commands are to be executed, it may be more convenient to write these statements into a file with Scilab editor.
\item<+-|alert@+> These are called as SCRIPT files.
\end{itemize}
\end{frame}

\begin{frame}[fragile]
\frametitle{Script Files}
\begin{itemize}
\item<+-|alert@+>  To execute the commands written in such a script file, the $"exec"$ function can be used, followed by the name of the script file.
\item<+-|alert@+>  These file generally have the extension $".sce"$ or $".sci"$, depending on its content.
\end{itemize}
\end{frame}

\begin{frame}[fragile]
\frametitle{.sci Files}
Files having the .sci extension contains\pause
\begin{itemize}
\item Scilab functions and/or
\item User defined functions
\end{itemize}\pause

\vspace{0.5cm}
Executing these files loads the functions into Scilab environment (but does not execute them)
\end{frame}

\begin{frame}[fragile]
\frametitle{.sce Files}
Files having the .sce extension contains\pause
\begin{itemize}
\item Scilab functions and
\item User defined functions
\end{itemize}
\end{frame}

\begin{frame}
\frametitle{Remember}
Please Remember that\pause
\vspace{0.5cm}
\begin{itemize}
\item The convention of naming the extension as $.sce$ and $.sci$ are not RULES, but a convention followed by the scilab community.
\end{itemize}
\end{frame}

\begin{frame}
\frametitle{Features of Scilab Script Files}
One of the interesting feature of scilab is\pause
\begin{itemize}
\item<+-|alert@+> You can define any number of functions in a single .sci file
\end{itemize}
\end{frame}

\begin{frame}
\frametitle{Features of Scilab Script Files}
\begin{itemize}
\item<+-|alert@+> While doing this please remember that
\begin{itemize}
\item<+-|alert@+> By default all the variables defined in a function are LOCAL
\item<+-|alert@+> The scope of these variables used in a particular function ends with the endfunction keyword of the function definition.
\end{itemize}
\end{itemize}
\end{frame}


\begin{frame}
\frametitle{Features of Scilab Script Files}
Advantage of this feature is\pause
\begin{itemize}
\item<+-|alert@+> We can use same variable names in different function.
\item<+-|alert@+> These variables won't get mixed up unless we use the global option.
\item<+-|alert@+> To know more about the global variables type $help$  $global$
\end{itemize}
\end{frame}

\begin{frame}[fragile]
\frametitle{Features of Scilab Script Files}

Please note that if any variable is to be "watched" or monitored inside a function, then $disp$ is required.\\\pause
\vspace{0.3cm}
Inside a function file, you can check for yourself the effect of putting a $semicolon ( ; )$ at the end of a statement\\\pause
\vspace{0.3cm}
 Also check this for $disp("..." )$ statements.

\end{frame}


\begin{frame}
\frametitle{Inline Functions}
\begin{itemize}
\item<+-|alert@+> Functions are segments of code that have well defined input and output as well as local variables
\item<+-|alert@+> The simplest way to define a function is by using the command $`deff()'$.
\end{itemize}
\end{frame}

\begin{frame}
\frametitle{Inline Functions}
\begin{itemize}
\item<+-|alert@+> Scilab allows the creation of in-line functions and are especially useful when the body of the function is short
\item<+-|alert@+> This can be done with the help of the function $deff()$.
\end{itemize}
\end{frame}

\begin{frame}[fragile]
\frametitle{Inline Functions}
\begin{itemize}
\item<+-|alert@+> It takes two string parameters.
\begin{itemize}
\item<+-|alert@+> The first string defines the interface to the function
\item<+-|alert@+> The second string defines the statements of the function.
\end{itemize}
\end{itemize}
\end{frame}

\begin{frame}[fragile]
\frametitle{Inline Functions}
\begin{itemize}
\item<+-|alert@+> The deff command defines the function in the scilab and also loads it.
\item<+-|alert@+> There is no need to load the function defined by using deff command explicitly through execute menu option.
\end{itemize}
\end{frame}

\begin{frame}
\frametitle{.sce Files}
The files with the .sce file extension are the script files
\begin{itemize}
\item<+-|alert@+> They contain the SCILAB commands that you enter during an interactive kind of SCILAB session
\item<+-|alert@+> They can comprise comment lines utilized in documenting the function
\end{itemize}
\end{frame}

\begin{frame}
\frametitle{.sce Files}
\begin{itemize}
\item<+-|alert@+> They can also use the command EXEC to execute the script.
\end{itemize}
\end{frame}

\begin{frame}
\frametitle{.sci Files}
\begin{itemize}
\item<+-|alert@+> The files with the .sci file extension are the function files that start with the function statement.
\end{itemize}
\end{frame}

\begin{frame}
\frametitle{.sce Files}
\begin{itemize}
\item<+-|alert@+> A single .sci file can have multiple function definitions which
\begin{itemize}
\item<+-|alert@+> Themselves contain any number of SCILAB statements that perform operations on the function arguments or
\item<+-|alert@+> On the output variables after they have been evaluated.
\end{itemize}
\end{itemize}
\end{frame}


% acknowledgement

\begin{frame}[fragile]
\frametitle{Acknowledgement}
\begin{itemize}
\item This spoken tutorial has been created by the Free and Open Source Software in Science and Engineering Education(FOSSEE).\pause
\item More information on the FOSSEE project could be obtained from {\color{magenta}http://fossee.in} or {\color{magenta}http://scilab.in}\pause
\end{itemize}
\end{frame}

\begin{frame}[fragile]
\frametitle{Acknowledgement}
\begin{itemize}
\item Supported by the National Mission on Eduction through ICT, MHRD, Government of India.\pause
\item For more information, visit: \\
	{\color{magenta}http://spoken-tutorial.org/NMEICT-Intro}

\end{itemize}
\end{frame}
\end{document} 

