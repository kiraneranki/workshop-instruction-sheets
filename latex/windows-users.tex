\documentclass[11pt]{article}
\title{Instructions to use Spoken Tutorials \\
\LaTeX\ for Linux Users}
\author{Kannan Moudgalya}
\date{28 July 2011}
\topmargin -0.75in
\textheight 9.5in
\textwidth 6.5in
\oddsidemargin 0in
\evensidemargin 0in
\usepackage{graphicx,multicol}
\newenvironment{enumcpt}{\begin{enumerate} \topsep 0pt \partopsep 0pt 
                        \parsep 0pt
                        \itemsep 0pt \leftmargin -1in \rightmargin 0pt
                        }{\end{enumerate}}

\pagestyle{empty}
\thispagestyle{empty}
\begin{document}
\begin{minipage}[t]{0.15\textwidth}
\includegraphics[width=\linewidth]{3t-logo}
\end{minipage} \hfill
\begin{minipage}[t]{0.68\textwidth}
\begin{center}
\vspace{-0.7in}
\Large
Spoken Tutorial Based \LaTeX\ Workshop: Windows \\
\large
Spoken Tutorial Team \\
IIT Bombay \\
5 August 2011
\end{center}
\end{minipage} \hfill
\begin{minipage}[t]{0.12\textwidth}
\includegraphics[width=\linewidth]{st-logo.jpg}
\end{minipage}



\begin{multicols}{2}

\section{The procedure to practise}
\begin{enumcpt}
\item You have been given a set of spoken tutorials and files
\item You will typically do one tutorial at a time
\item You may Listen to a spoken tutorial and reproduce all the
  commands shown in the video
\item If you find it difficult to do the above, you may consider
  listening to \emph{whole} tutorial once and then practise during
  the second hearing
\end{enumcpt}

\section{First tutorial: {\tt What is
    Compilation?}}
\begin{enumerate}
\item Please locate the folder {\tt LaTeX\_Workshop} on the CD and
  copy it to your Desktop, this folder contains all the tutorials,
  please locate {\tt LaTeX-Win.wmv} in folder {\tt 02-MiKTeX}
\item Please listen to the video {\tt LaTeX-Win.wmv} until 1:04
  minute, when there is a mention of {\tt What is Compilation?}.
  Pause this video at this time.
\item Listen to the video {\tt compiling.mov} which is available in
  folder {\tt 01-compilation} - just listen once.  Do not attempt to
  re-do it on MiKTeX yet.
\item Continue with {\tt LaTeX-Win.wmv} from where you left.  Follow
  the installation procedure for MiKTeX and TeXNicCenter.  As the
  setup files for installing MiKTeX and TeXNicCenter are already
  available, you do not need to download them, but only install them.
  Please install them as suggested in this video.
\item If Adobe reader is already installed on your system, pause
  {\tt LaTeX-Win.wmv} at 10:52min only.  You would now have MiKTeX,
  TeXNicCenter and Adobe reader, configured as in this video.  
\item Please locate the file {\tt hello.tex} in folder {\tt
  01-compilation}.
\item You may now repeat whatever the video {\tt LaTeX-Win.wmv} says
  from 10:52min to 16:12min.
\item If the Adobe reader is not installed on your system, you may
  consider installing {\tt sumatra} pdf reader, as per the
  instructions given in the video {\tt LaTeX-Win.wmv}, from 20:00min
  till the end of this video.  Follow only the sumatra installation
  part - the source file is given to you. You only need to
  \emph{listen} to other instructions.  For example, do not attempt to
  work with letter.tex or report.tex yet.
\end{enumerate}

\section{Procedure for letter writing}
\begin{enumcpt}
\item For this tutorial, you will need the files {\tt letter.tex} and
  {\tt letter.mov} from {\tt 03-letter}, which is a sub-folder of {\tt
    LaTeX\_Workshop} that is available on Desktop.
\item For this tutorial, you will listen to {\tt letter.mov}.  Once
  again, you have to open using VLC.  
\item Repeat all the instructions given for {\tt What is
    Compilation?}.  Remember to do the following:
\begin{itemize}
\item Substitute {\tt hello} with {\tt letter} in every instruction.
  For example, you shall copy the file {\tt letter.tex} into your home
  folder, as an initial step.
  \end{itemize}
\item Do not attempt to create the file {\tt letter.tex} from
  scratch.  You are likely to make mistakes.
\end{enumcpt}

\section{After letter writing, what to do?}
After {\tt letter writing}, you will do the following tutorials, in
this order.  For each tutorial, the name of the sub-folder, the video
to watch and the files to copy to your home folder, are shown below:
\begin{description}
\item [Report writing:]
Sub-folder name: {\tt 04-report}.
File to be copied to the home folder: {\tt report.tex}.  Video to
watch: {\tt report.mov}. 
\item [Maths:]
Sub-folder name: {\tt 05-maths}. \\
File to copy: {\tt maths.tex}.  Video to watch: {\tt maths.mov}.
\item [Equations:]
Sub-folder name: {\tt 06-equations}. \\
File to copy: {\tt equations.tex}.  Video to watch: {\tt
  equations.mov} and possibly, {\tt MiKTeX-update.wmv} in
  folder {\tt 02-MiKTeX}.
\begin{enumcpt}
\item If your system has only a basic MiKTeX installation, you may
  get an error message that {\tt cclicenses.sty} is not found or
  something like that.
\item The above will happen if a package is not installed.  To install
  the missing packages, please listen to {\tt MiKTeX-update.wmv} in
  folder {\tt 02-MiKTeX} and follow the instructions.
  If this also fails, only then, do an Internet
  search, locate the missing file and download to your working folder
  - in this case, your {\tt home} folder.  You may have to do this for
  the file {\tt cclicenses.sty}.
\item We have provided the file {\tt cclicenses.sty} in sub-folder {\tt
    06-equations} for your convenience.
\item From now on, you should follow the procedure discussed in the
  above two bulleted points whenever a package is
  missing.
\item Once you become more comfortable with \LaTeX, you will learn
  about a central location where you can copy the missing packages.
  Until that happens, copy the missing packages into your working
  folder. 
\end{enumcpt}
\item [Tables and Figures:]
Sub-folder name: {\tt 07-table-figure}. 
Files to copy: {\tt tab-fig.tex}, {\tt
  iitb.pdf} and {\tt iitblogo.pdf}.  In addition, you may need other
files from the previous tutorials, as shown in the spoken tutorial.
Video: {\tt tab-fig.mov}.
\begin{itemize}
\item In case of any warning, such as {\tt cclicenses.sty} not found,
  please see the instructions for the tutorial {\tt equations}, given
  above and install the missing packages.
\end{itemize}
\item [Bibliography:] 
Sub-folder name: {\tt 08-references}.  Video to watch: {\tt
  references.mov}.  
Files to copy: {\tt references.tex}, {\tt
  harvard.sty}, {\tt ifac.bst} and {\tt ref.bib}.
\begin{itemize}
\item In case of any warning that some package is not found,
  please see the instructions for the tutorial {\tt equations}, given
  above and install the missing packages.
\end{itemize}
\item [Beamer]
Sub-folder name: {\tt 09-beamer}. \\
Files to be copied: {\tt beamer.tex}, {\tt
  iitb.pdf}, {\tt iitblogo.pdf} and any other files, as mentioned in
the spoken tutorial.  Video to watch: {\tt beamer.mov}.
\begin{itemize}
\item In case of any warning, such as {\tt beamer} is not found,
  please see the instructions for the tutorial {\tt equations}, given
  above and install the missing packages.
\end{itemize}
\end{description}
\end{multicols}
\end{document}


\subsection*{When to Go to the Next Tutorial?}
\begin{itemize}
\item You should be able to reproduce all the commands shown in the
  video. 
\item Some \emph{new} things that you may try may or may not work.
\item You cannot learn all of \LaTeX\ in two hours.
\item Instead of wasting a lot of time on things that do not work,
  please spend time in listening to things that work, i.e. after a
  little bit of experimentation with a tutorial, please go to the next
  tutorial.
\item The workshop organisers may not know \LaTeX\ commands. Their
  role is only to conduct this workshop using spoken tutorials.
\end{itemize}

\subsection*{Procedure for other tutorials}
\begin{enumerate}
\item You are expected to listen to the \LaTeX\ tutorials in the following
order:
\begin{enumerate}
\item LaTeX for Windows
\item What is compilation
\item Letter writing
\item Report writing
\item Maths
\item Equations
\item Tables and Figures
\item Bibliography
\item Beamer
\end{enumerate}
For example, if you just completed {\tt What is compilation?}, it is
time to start working on letter writing, using the video {\tt letter.mov}.

\item All the required source files have been made available to you.
  Please copy ALL the source files used in the tutorial to your
  Desktop and work with them.  Please do NOT create them from scratch,
  as you may make mistakes.


\begin{quote}
Please do NOT create them from scratch, as you may make
mistakes.
\end{quote}

\item Carry out all the steps explained for the {\tt What is
    Compilation?} tutorial now also.  For example, follow all of the
  above steps for letter writing using the file name {\tt letter.tex},
  as opposed to {\tt hello.tex}.
\end{enumerate}

\subsection*{Installation}
Please follow these instructions carefully, else you may waste a lot
of time.
\begin{itemize}
\item We advise you to install MiKTeX on your windows systems using
  instructions explained in the tutorial, available at {\tt
    http://spoken-tutorial.org/LaTeX\_Windows\_English} only.
\item In view of this, it may be better for you to un-install MiKTeX
  and re-install from scratch using this video instructions.
\item Suppose that you do not follow the above, but persist with your
  earlier installation.  In case you run into any difficulty, such as,
  file not found, etc., you must un-install MiKTeX and re-install,
  using the instructions given above.  The workshop organisers will
  not be able to help you fix the path problem.
\item We are ready to help you install MiKTeX ahead of the workshop,
  if you visit our lab in room 311, automation lab, second floor,
  chemical engineering building.  In case you have any question, you
  may contact Shahid at 96191 16925 or Sachin at 99204 88086.
\end{itemize}

\end{multicols}
\end{document}
