\documentclass[11pt]{article}
\title{Instructions to use Spoken Tutorials \\
\LaTeX\ for Linux Users}
\author{Kannan Moudgalya}
\date{28 July 2011}
\topmargin -1.25in
\textheight 10.5in
\textwidth 6.5in
\oddsidemargin 0in
\evensidemargin 0in
\usepackage{graphicx,multicol}
\newenvironment{enumcpt}{\begin{enumerate} \topsep 0pt \partopsep 0pt 
                        \parsep 0pt
                        \itemsep 0pt \leftmargin -1in \rightmargin 0pt
                        }{\end{enumerate}}

\pagestyle{empty}
\thispagestyle{empty}
\begin{document}
\begin{minipage}[t]{0.15\textwidth}
\includegraphics[width=\linewidth]{3t-logo}
\end{minipage} \hfill
\begin{minipage}[t]{0.65\textwidth}
\begin{center}
\vspace{-0.7in}
\Large
Spoken Tutorial Based \LaTeX\ Workshop on Linux \\
\large
Spoken Tutorial Team \\
IIT Bombay \\
5 August 2011
\end{center}
\end{minipage} \hfill
\begin{minipage}[t]{0.12\textwidth}
\includegraphics[width=\linewidth]{st-logo.jpg}
\end{minipage}



\begin{multicols}{2}

\section{The procedure to practise}
\begin{enumcpt}
\item You have been given a set of spoken tutorials and files
\item You will typically do one tutorial at a time
\item You may Listen to a spoken tutorial and reproduce all the
  commands shown in the video
\item If you find it difficult to do the above, you may consider
  listening to \emph{whole} tutorial once and then practise during
  the second hearing
\end{enumcpt}

\section{First tutorial: {\tt What is
    Compilation?}}
These detailed instructions are intended mainly for the Windows users,
who may have to use Linux for learning \LaTeX.  The Linux users will
know most of these things.
\begin{enumcpt}
\item Click {\tt Places} button at
  the top left hand corner and then click the {\tt Home Folder}.  The
  folder that opens is called your {\tt home} folder.
\item Please locate the folder {\tt LaTeX\_Workshop} that is available
  on Desktop.  The sub-folder {\tt 01-compilation} contains the
  following files that you need for this tutorial: {\tt hello.tex} and
  {\tt compiling.mov}.
\item Please copy {\tt hello.tex} from this folder to your {\tt home}
  folder. 
\item Open the {\tt terminal} using the command {\tt Ctrl-Alt-t}, by
  pressing all these three keys simultaneously.
\item Open the file that you copied above into the editor using the
  command
\begin{quote}
  {\tt gedit hello.tex~\&}  
\end{quote}
Do not forget the symbol ampersand (\&) at the end of the command,
obtained by pressing {\tt shift 7}.  Please leave spaces exactly as
given above.
\item Right click on {\tt compiling.mov}, point the cursor on  {\tt Open With} and select {\tt VLC Media
    Player}, now listen to this spoken tutorial.
\item As shown in the video at 1:57min, compile from the terminal the
  file {\tt hello.tex} using the command  
\begin{quote}
  {\tt pdflatex hello.tex}
\end{quote}
Note that {\tt pdflatex} is ONE command.  Please do not leave a space
between pdf and latex.
\item Pause the video at 2:04min.  You should now be able to give the
  command {\tt pdflatex hello.tex} and get a file {\tt hello.pdf}.  If 
  there is any difficulty in this step, please listen to the tutorial
  from 1:57min to 2:04min once again.
\item The video talks about a pdf viewer called {\tt skim} at
  3:04min.  
\begin{itemize}
\item Please do not attempt to use {\tt skim} - it is NOT available on
Linux.
\end{itemize}
You have to use the pdf viewer {\tt evince} instead.  Give the
following command from the {\tt terminal} to open the pdf file:
\begin{quote}
  {\tt evince hello.pdf \&}
\end{quote}
Once again, do not forget the \& symbol in the above command.

\item Observe that there are three activities in the video:
  \begin{enumcpt}
  \item Editing the file hello.tex
  \item Compiling this file on the terminal
  \item Viewing the file in a pdf viewer.
  \end{enumcpt}
  You have now carried out all of these three activities through {\tt
    gedit}, {\tt terminal} and {\tt evince}, respectively.  You may
  wish to resize and arrange these three screens so that you can
  access all of them simultaneously.  They could be overlapping, of
  course.  You should be able to reproduce every one of these three
  activities, as shown in the spoken tutorial.
\item From now on, you are supposed to do this: 
\begin{quote}
Listen to a command, pause the video, and try to reproduce it
\end{quote}
\item Please note that you will NOT have to open {\tt gedit} and {\tt
    evince} again.  You will have to repeatedly give the {\tt
    pdflatex} command, however, as explained above.  This is what is
  shown in the spoken tutorial also.
\item It is possible that some changes that you try may create
  problems during the {\tt pdflatex} command.  If this happens, type
  the letter {\tt x} or {\tt ctrl-d} in the terminal to come out.
  Until you become comfortable with \LaTeX\, please try only the
  commands shown in the spoken tutorial.
\item After reproducing all the commands, please go to the next
  tutorial, {\tt letter writing}.
\end{enumcpt}

\section{Procedure for letter writing}
\begin{enumcpt}
\item Close {\tt gedit}, {\tt evince} screens and the {\tt terminal}
  that you used for the previous tutorial. 
\item For this tutorial, you will need the files {\tt letter.tex} and
  {\tt letter.mov} from {\tt 03-letter}, which is a sub-folder of {\tt
    LaTeX\_Workshop} that is available on Desktop.
\item For this tutorial, you will listen to {\tt letter.mov}.  Once
  again, you have to open using VLC.  
\item Repeat all the instructions given for {\tt What is
    Compilation?}.  Remember to do the following:
\begin{itemize}
\item    Substitute {\tt hello} with {\tt letter} in every instruction.
    For example, you shall copy the file {\tt letter.tex} into your
    home folder, as an initial step.
  \end{itemize}
\item Do not attempt to create the file {\tt letter.tex} from
  scratch.  You are likely to make mistakes.
\end{enumcpt}

\section{After letter writing, what to do?}
After {\tt letter writing}, you will do the following tutorials, in
this order.  For each tutorial, the name of the sub-folder, the video
to watch and the files to copy to your home folder, are shown below:
\begin{description}
\item [Report writing:]
Sub-folder name: {\tt 04-report}.
File to be copied to the home folder: {\tt report.tex}.  Video to
watch: {\tt report.mov}. 
\item [Maths:]
Sub-folder name: {\tt 05-maths}. \\
File to copy: {\tt maths.tex}.  Video to watch: {\tt maths.mov}.
\item [Equations:]
Sub-folder name: {\tt 06-equations}. \\
File to copy: {\tt equations.tex}.  Video to watch: {\tt equations.mov}.
\begin{enumcpt}
\item If your system has only a basic \LaTeX\ installation, you may
  get an error message that {\tt cclicenses.sty} not found or a
  similar warning.
\item The above will happen if a package is not installed.  Please install the package {\tt texlive-full} using the {\tt Synaptic
    Package Manager}.  If this also fails, only then, do an Internet
  search, locate the missing file and download to your working folder
  - in this case, your {\tt home} folder.  You may have to do this for
  the file {\tt cclicenses.sty}.
\\
We have provided the file {\tt cclicenses.sty} in sub-folder {\tt 06-equations} for your convenience.
\item From now on, you should follow this procedure whenever a file is
  missing.
\item Once you become more comfortable with \LaTeX, you will learn
  about a central location where you can copy the missing packages.
  Until that happens, copy the missing packages into your working
  folder. 
\end{enumcpt}
\item [Tables and Figures:]
Sub-folder name: {\tt 07-table-figure}. 
Files to copy: {\tt tab-fig.tex}, {\tt
  iitb.pdf} and {\tt iitblogo.pdf}.  In addition, you may need other
files from the previous tutorials, as shown in the spoken tutorial.
Video: {\tt tab-fig.mov}.
\begin{itemize}
\item In case of any warning, such as {\tt cclicenses.sty} not found,
  please see the instructions for the tutorial {\tt equations}, given
  above.
\end{itemize}
\item [Bibliography:] 
Sub-folder name: {\tt 08-references}.  Video to watch: {\tt
  references.mov}.  
Files to copy: {\tt references.tex}, {\tt
  harvard.sty}, {\tt ifac.bst} and {\tt ref.bib}.
\item [Beamer]
Sub-folder name: {\tt 09-beamer}. \\
Files to be copied: {\tt beamer.tex}, {\tt
  iitb.pdf}, {\tt iitblogo.pdf} and any other files, as mentioned in
the spoken tutorial.  Video to watch: {\tt beamer.mov}.
\end{description}
\end{multicols}


\end{document}
\section{The software that you will need are:}
\pagestyle{empty}
\thispagestyle{empty}
\begin{enumcpt}
\item \LaTeX\ itself
\item A pdf viewer, evince could work
\item An editor, for example, gedit
\item VLC player
\end{enumcpt}


\section{The sequence in which you will practise the tutorials}
\begin{enumcpt}
\item What is compilation
\item Letter writing
\end{enumcpt}

\begin{itemize}
\item Please reproduce all the commands shown in the spoken tutorial.
\item Initially try only the commands shown in the spoken tutorial.

\item Some {\emph new} things that you may try may or may not work.
\item You cannot learn all of \LaTeX\ in two hours.
\item Instead of wasting a lot of time on things that do not work,
  please spend time in listening to things that work, i.e. after a
  little bit of experimentation with a tutorial, please go to the next
  tutorial.
\item The workshop organisers may not know \LaTeX\ commands. Their role is only to conduct this workshop using spoken tutorials.
\end{itemize}