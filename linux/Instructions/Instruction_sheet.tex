\documentclass[11pt]{article}
\title{Instructions to use Spoken Tutorials \\
\LaTeX\ for Linux Users}
\author{Kannan Moudgalya}
\date{28 July 2011}
\topmargin -1.25in
\textheight 10.5in
\textwidth 6.5in
\oddsidemargin 0in
\evensidemargin 0in
\usepackage{graphicx,multicol}
\newenvironment{enumcpt}{\begin{enumerate} \topsep 0pt \partopsep 0pt 
                        \parsep 0pt
                        \itemsep 0pt \leftmargin -1in \rightmargin 0pt
                        }{\end{enumerate}}

\pagestyle{empty}
\thispagestyle{empty}
\begin{document}
\begin{minipage}[t]{0.15\textwidth}
\includegraphics[width=\linewidth]{3t-logo}
\end{minipage} \hfill
\begin{minipage}[t]{0.65\textwidth}
\begin{center}
\vspace{-0.7in}
\Large
Spoken Tutorial Based Linux Workshop \\
\large
Spoken Tutorial Team \\
IIT Bombay \\
10 August 2011
\end{center}
\end{minipage} \hfill
\begin{minipage}[t]{0.12\textwidth}
\includegraphics[width=\linewidth]{st-logo.jpg}
\end{minipage}


\begin{multicols}{2}

\textbf {The procedure to practise}
\begin{enumcpt}
\item You have been given a set of Spoken tutorials and files.
\item You will typically do one tutorial at a time.
\item You may Listen to a Spoken tutorial and reproduce all the
  commands shown in the video.
\item If you find it difficult to do the above, you may consider
  listening to \emph{whole} tutorial once and then practise during
  the second hearing.
\item You must go through the Spoken tutorials in following sequence.
\end{enumcpt}

\section{Ubuntu Desktop:}
Tutorial required: {\tt 01-ubuntu-desktop.ogv}

\begin{enumcpt}
\item Please locate the folder {\tt Linux\_Workshop} that is available
  on your Desktop, this folder contains all the Linux tutorials.
\item Please locate the tutorial \\ {\tt 01-ubuntu-desktop.ogv}
\item Right click on {\tt 01-ubuntu-desktop.ogv}, point the cursor on
  {\tt Open With} and select {\tt VLC Media Player}, now listen to
  this Spoken tutorial.
\item Please follow the tutorial as show in video.
\end{enumcpt}

 In this tutorial you will learn about Ubuntu Operating System which
 uses Gnome Desktop Enviroment. Applications, Places and System menus
 are very well covered in this tutorial. Introduction to calculator,
 text-editor, Gnome-terminal, firefox web browser and Office
 application suite are covered in this tutorial. You will also get an
 idea about the {\tt Places} and various options under the {\tt
   System} menus in this tutorial.

\section{Synaptic Package Manager:}
Tutorial required: {\tt 02-synaptic.ogv} 

\begin{enumcpt}
\item Please follow the tutorial as shown in video.
\end{enumcpt}

 Synaptic Package Manager is a tool to install Software packages. In
 this tutorial you will learn how to configure repositories, network
 proxy and install/remove software packages in Ubuntu Operating
 System.

\section{Basic Linux Command:}
Tutorial required: {\tt 03-basic-commands.ogv} 

\begin{enumcpt}
\item Close all the application that you have opened.
\item Open terminal by pressing {\tt Ctrl-Alt-t} keys simultaneously.
\item Please follow the tutorial as shown in video.
\item After reproducing all the commands. please go to the next
  tutorial \textbf {General Purpose Utilities}.
\end{enumcpt}

This tutorial covers the basic commands in Linux. In this tutorial you
will learn about linux commands and a command interpreter. Also you will
learn about various type of shells, how to access manuals and get
help on terminal using {\tt man} command.

\section{General Purpose Utilities in Linux:}
Tutorial required: {\tt 04-gen-purpose-utils.ogv} 

\begin{enumcpt}
\item Please follow the tutorial as shown in video.
\end{enumcpt}

In this tutorial you will learn some of the most basic yet heavily
used commands on Linux. The main motivation of this tutorial is to
give you a head start about working with Linux. This tutorial covers
{\tt echo}, {\tt uname}, {\tt whoami}, {\tt passwd}, {\tt date}, {\tt
  cal}, {\tt pwd}, {\tt ls},and {\tt cat} utilities available in
Linux.

\section{Linux File System:}
Tutorial required: {\tt 05-linux-file-system.ogv} 
\begin{enumcpt}
\item Please follow the tutorial as shown in video.
\end{enumcpt}
This tutorial is about the various file systems in Linux. The
tutorial explains what are regular files, device file and folder in
Linux. Current and parents directries is very well explained in this
tutorial. Linux commands like {\tt echo}, {\tt pwd}, {\tt cd}, {\tt
  mkdir}, and {\tt rmdir} are covered in this tutorial.

\section{File Attributes:}
Tutorial required: {\tt 06-file-attributes.ogv} 

\begin{enumcpt}
\item Open terminal by pressing {\tt Ctrl-Alt-t} keys simultaneously.
\item Please create an empty file {\tt example1} using the following
  command in the terminal.
\begin{quote}
{\tt touch example1}
\end{quote}
\item Similarly create the following empty files also: {\tt example1,
example2, example3, example4, example5} and {\tt testchown}
\item Please follow the tutorial as shown in video.
\end{enumcpt}

This tutorial is about the Linux File Attributes. This tutorial
describes file permissions, ownership of a file, group permissions,
inode, soft and hard links. In this tutorial you will learn how to
deal with file/folder permissions, ownership and group
permissions. This tutorial will also cover inode number of a file, and
how to create a soft and hard links. You will also learn commands such
as {\tt chown}, {\tt chmod}, {\tt chgrp}, and {\tt ln}.

\section{Redirections and Pipes:}
Tutorial required: {\tt 07-redirection-pipes.ogv} 

\begin{enumcpt}
\item Please follow the tutorial as shown in video.
\end{enumcpt}

This tutorial covers the standard input/output, standard error,
redirection, pipes, and file descriptor. With pipes and redirection,
one can ``chain'' multiple programs to become extremely powerful
commands. Most programs on the command-line accept different modes of
operation. Many can read and write to files for data, and most can
accept standard input or output.

\section{Working with Linux Process:}
Tutorial required: {\tt 08-linux-process.ogv} 

\begin{enumcpt}
\item Please follow the tutorial as shown in video.
\end{enumcpt}

This tutorial contains the very rich material on Linux Process, Shell
Process, Process ID, Parent Process ID, Spawning, ps, parent, child,
subshell, system process, init etc. The Process is one of the
fundamental abstractions in Linux Systems, the other fundamental
abstraction being files. A process consists of the executing program
code, a set of resources such as open files, internal kernel data, an
address space, one or more threads of execution and a data section
containing global variables.

\section{The Linux Environment:}
Tutorial required: {\tt 09-linux-environment.ogv}

\begin{enumcpt}
\item Please follow the tutorial as shown in video.
\end{enumcpt}

This tutorial is about the Linux environment which determines how the
operating system behaves with you, how it responds to your commands,
how it interprets your actions and so on. Linux can be highly
customised by changing the settings of the shell. This tutorial
covered the environment and local variables, {\tt PATH}, {\tt HOME},
profile, {\tt history}, and alias.

\section{Simple Filter:}
Tutorial required: {\tt 10-simple-filter.ogv} 

\begin{enumcpt}
\item Please follow the tutorial as shown in video.
\end{enumcpt}

A filter takes the standard input, does something useful with it, and
then returns it as a standard output. Linux has large number of
filters. In this tutorial covers commands {\tt head}, {\tt tail}, {\tt
  sort}, {\tt cut}, {\tt paste}, {\tt grep}, and {\tt sed}.


\end{multicols}


\end{document}
\section{The software that you will need are:}
\pagestyle{empty}
\thispagestyle{empty}

