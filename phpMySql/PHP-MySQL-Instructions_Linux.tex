\documentclass[11pt]{article}
\title{Instructions to use Spoken Tutorials \\
  PHP-MySQL for Linux Users}
\author{Shahid Farooqui}
\date{07 October 2011}
\topmargin -1.25in
\textheight 10.5in
\textwidth 6.5in
\oddsidemargin 0in
\evensidemargin 0in
\usepackage{graphicx,multicol,amsmath}
\newenvironment{enumcpt}{\begin{enumerate} \topsep 0pt \partopsep 0pt 
    \parsep 0pt
    \itemsep 0pt \leftmargin -1in \rightmargin 0pt
}{\end{enumerate}}

\pagestyle{empty}
\thispagestyle{empty}
\begin{document}
\begin{minipage}[t]{0.15\textwidth}
  \includegraphics[width=\linewidth]{3t-logo}
\end{minipage} \hfill
\begin{minipage}[t]{0.65\textwidth}
  \begin{center}
    \vspace{-0.7in}
    \Large
    Instruction Sheet for PHP \& MySQL \\
    Spoken Tutorials in Ubuntu Linux OS \\
    \large
    Spoken Tutorial Team \\
    IIT Bombay \\
  \end{center}
\end{minipage} \hfill
\begin{minipage}[t]{0.12\textwidth}
  \includegraphics[width=\linewidth]{st-logo.jpg}
\end{minipage}
\begin{multicols}{2}

  \section{The procedure to practice}
  \begin{enumcpt}
  \item You have been given a set of spoken tutorials and files
  \item You will typically do one tutorial at a time
  \item You may listen to a spoken tutorial and reproduce all the commands shown in the video
  \item If you find it difficult to do the above, you may consider
    listening to the \emph{whole} tutorial once and then practise during
    the second hearing
  \end{enumcpt}

  \section{First tutorial: Installing a Webserver with PHP and MySQL (XAMPP in Windows)}
  \begin{enumcpt}
  \item Please locate the folder {\tt phpandmysql} and open it. This folder
    contains all the required materials.
  \item Double click on {\tt php\_tutorial.html} file to open it in the
    browser.  This displays the suggested learning sequence of the
    tutorials.
  \item Locate the topic \emph{Installing a Webserver with PHP and
    MySQL}. Click on the link \emph{Spoken Tutorial} to open and view the
    video.  Links have been provided for viewing the Script and Codes,
    as well.  You are not required to view the script unless you have
    difficulty in understanding the pronunciation or to refer to a
    command in the video.  You may copy-paste the given codes for your
    practice.
  \item Please listen to this tutorial and follow all the instructions
    carefully. From 1:19 min to 1:28 min the tutorial shows how to
    download XAMPP. We are assuming that XAMPP is already installed on
    your machine.  If you do not know how to install, please refer to
    the instructions given in {\tt XAMPP\_installation\_linux.pdf} in the
    {\tt phpandmysql} folder.
  \item At 6:34 the video talks about the \textbf{ConTEXT} text editor. But for
    Ubuntu linux, the eqvivalent text editor is {\tt Gedit}. You can use Gedit
    as your default text editor throughout the tutorials.
  \item Open a new tab in the firefox web-browser by pressing {\tt ctrl-t} or
    open a new web-browser. Type the URL {\tt http://localhost/phpacademy/} in
    your address bar. Now you will be able to run all your {\tt .php} and
    {\tt .html} files using this URL.
  \item Please note that the path of phpacademy folder shown in video is
    {\tt c:$\backslash$xampp$\backslash$htdocs$\backslash$phpacademy}
    this is your working directory in windows. But for Ubuntu Linux
    equivalent path is: {\tt /opt/lampp/htdocs/phpacademy}. This will be your
    working directory.  Henceforth, for all the videos, the {\tt .php} and/or
    {\tt .html} files should be created/copied in this directory. You are free
    to create subdirectories here for each video tutorial so that you
    can manage all your files in a better way.

    \textbf{Note}: Users who are new to Ubuntu Linux, please go through
    the tutorial {\tt ubuntu\_desktop.ogv} given in the {\tt phpandmysql} folder. This
    tutorial explains how to open a terminal window and how to use {\tt gedit}
    text editor.
  \end{enumcpt}

  \section{Second tutorial: Echo Function}
  \begin{enumcpt}
  \item Create the file {\tt helloworld.php} in the folder {\tt
    /opt/lampp/htdocs/phpacademy} as it is required for this video
    tutorial. To do this:
    \begin{enumcpt}
    \item Open the terminal using the command {\tt Ctrl-Alt-t} by
      pressing all these three keys simultaneously.
    \item Now type
      \begin{quote}
        {\tt cd /opt/lampp/htdocs/phpacademy}
      \end{quote}
      in the terminal and hit {\tt ENTER} key.
    \item  Now type 
      \begin{quote}
        {\tt gedit helloworld.php \&}
      \end{quote}
      and hit {\tt ENTER} key.
    \end{enumcpt}

  \item Now go back to {\tt php\_tutorial.html} inside {\tt phpandmysql}
    folder and locate the topic \emph{Echo function} and click on the
    link \emph{Spoken Tutorial} to open and view the video. Please
    follow the instructions in the video.
  \item Type all the code shown in the video in {\tt helloworld.php}
    file and save periodically by clicking \textbf{File} and
    \textbf{Save}.
  \item At time 1:07 min, the tutorial shows Firefox Web Browser to view
    {\tt helloworld.php} file, you can view this file in seperate TAB or
    in a new web-browser.
  \item Type {\tt http://localhost/phpacademy/} in the address bar of
    your firefox browser. This will show a list of files. Click {\tt
      helloworld.php}. This will open {\tt helloworld.php} in the
    browser.
  \item Now every time you make some change to {\tt helloworld.php}
    using {\tt gedit} editor, you need to save your changes and refresh
    your web browser by pressing {\tt F5} key to reflect the changes.
  \item In some of the future tutorials \emph{Google Chrome} is used as
    the web browser, but you can continue using Firefox or any other web
    browser.
  \item From time 1:55 min, the tutorial says about {\tt parse
    error}. Please understand it carefully and try to reproduce the
    exact code as shown in the tutorial.
  \item After you are done with this tutorial, you can move on to our
    next topic, \emph{Variables in PHP} in {\tt php\_tutorial.html}.
  \end{enumcpt}

  \section{Similarly, remember to follow these three simple steps for all the tutorials - }
  \begin{enumcpt}
  \item Go to the next topic in the {\tt php\_tutorial.html}. Click on {\tt Spoken Tutorial} link to view the video.
  \item Create new file with same name as that mentioned in the video tutorial in {\tt /opt/lampp/htdocs/phpcacademy} using gedit text editor.
  \item Execute your PHP code in web browser.
  \end{enumcpt}

\end{multicols}
\end{document}


