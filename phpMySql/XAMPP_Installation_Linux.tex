\documentclass[11pt]{article}
\title{Instructions to use Spoken Tutorials \\
PHP-MySQL for Linux Users}
\author{Shahid Farooqui}
\date{7 September 2011}
\topmargin -1.25in
\textheight 10.5in
\textwidth 6.5in
\oddsidemargin 0in
\evensidemargin 0in
\usepackage{graphicx,multicol,amsmath}
\newenvironment{enumcpt}{\begin{enumerate} \topsep 0pt \partopsep 0pt 
                        \parsep 0pt
                        \itemsep 0pt \leftmargin -1in \rightmargin 0pt
                        }{\end{enumerate}}

\pagestyle{empty}
\thispagestyle{empty}
\begin{document}
\begin{center}
\line(1,0){250}
\end{center}
\begin{center}
\textbf {Instructions for XAMPP Installation}
\end{center}
\begin{multicols}{2}
{\tt XAMPP} is an easy to install Apache distribution containing MySQL, PHP and Perl. XAMPP is really very easy to install and to use - just download, extract and start. 

XAMPP is available for the following OS:

{\tt XAMPP} for {\tt Linux}

{\tt XAMPP} for {\tt Windows}

{\tt XAMPP} for {\tt Mac OSX}

{\tt XAMPP} for {\tt Solaris}

\textbf {XAMPP for Linux}

In the past this software was called {\tt LAMPP} but to avoid misconceptions it was renamed {\tt XAMPP} for {\tt Linux}.
 
\textbf {Installation in 6 Steps-}
\begin{enumcpt}
\item Locate this file"

Locate {\tt xampp-linux-1.7.4.tar.gz} inside the CD or the folder in your system.

\item Installation:

Simply type in the following commands:
Go to a Linux shell and login as the system administrator root:

  {\tt su}

Extract the downloaded archive file to /opt:

  {\tt tar xvfz xampp-linux-1.7.4.tar.gz -C /opt}

{\tt XAMPP} is now installed below the {\tt /opt/lampp} directory. 

\item Start:

To start {\tt XAMPP} simply call this command:

{\tt /opt/lampp/lampp start}

If you get any error messages please take a look at the Linux FAQ. 

\item Test:

To check that everything really works, just type in the following URL in your web browser: 

{\tt http://localhost}

You will see {\tt http://localhost/xampp} in your web browser URL.  This shows that the Apache server is running.

\item Create a directory called {\tt phpacademy} inside the 
{\tt /opt/lampp/htdocs} folder using the command 

{\tiny sudo mkdir -p /opt/lampp/htdocs/phpacademy}

\item Change the access permission of the directory phpacademy in order to run the php files.  To do so, type the command 

{\tiny sudo chmod 777 -R /opt/lampp/htdocs/phpacademy}
\end{enumcpt}



\textbf {Final Check-}
\begin{enumcpt}
\item All example code files that you create while learning should be kept in the {\tt phpacademy} folder only.

\item Now go to your browser and type the URL 
{\tt http://localhost/phpacademy/}

Now you will be able to run your .php files.  
\end{enumcpt}
For more detailed information, please visit

{\tiny http://www.apachefriends.org/en/xampp-linux.html}


For information on how to install {\tt XAMPP} on other OS, please visit the following URLs-

\emph {XAMPP for Windows}

{\tiny http://www.apachefriends.org/en/xampp-windows.html}

\emph {XAMPP for Mac OSX}

{\tiny http://www.apachefriends.org/en/xampp-macosx.html}

\emph {XAMPP for Solaris}

{\tiny http://www.apachefriends.org/en/xampp-solaris.html}

\end{multicols}
\end{document}
